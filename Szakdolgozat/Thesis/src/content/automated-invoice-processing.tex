\chapter{Automatizált Számlafeldolgozási Prototípus Tervezése és Megvalósítása}
\label{chap:megvalositas}

\section{Bevezetés: A Prototípus Céljai és Keretrendszere}
\label{sec:prototype_intro}
% Ebben a fejezetben kerül bemutatásra a ...
% Hivatkozás a korábbi fejezetekben (pl. \ref{sec:problemfeltaras_hibai_koltsegei}) azonosított problémákra.

\section{A Prototípus Architektúrája és Komponensei}
\label{sec:prototype_architecture}
% Az architektúra felépítése, hogyan kapcsolódik a BPA, a DOX, a Work Zone és az SAP.

\subsection{Adatkinyerés Konfigurálása (Document Information Extraction)}
\label{ssec:prototype_dox}
% A DOX szolgáltatás beállítása, a séma (schema) és a sablon (template) létrehozása.

\subsection{A Jóváhagyási Folyamat Tervezése (Process Builder)}
\label{ssec:prototype_process_builder}
% A BPMN 2.0 folyamatábra felépítése a Process Builderben.
% Lépések: Trigger, DOX hívás, Döntési tábla hívás, Emberi feladat (Forms), Bot hívás.

\subsection{Üzleti Szabályok Definiálása (Decisions)}
\label{ssec:prototype_decisions}
% A jóváhagyási mátrix (döntési tábla) létrehozása.
% Pl. Összeghatár alapján ki a jóváhagyó.

\subsection{Felhasználói Felületek Létrehozása (Forms és Work Zone)}
\label{ssec:prototype_ui}
% A jóváhagyási űrlap (Form) megtervezése.
% Az űrlap integrálása az SAP Build Work Zone "My Inbox" (Task Center) felületébe.

\subsection{Integráció az SAP Rendszerrel (RPA Bot)}
\label{ssec:prototype_rpa}
% Az "unattended" (felügyelet nélküli) bot felépítése.
% A bot lépései: Bejelentkezés SAP-ba, FB60/FV60 tranzakció futtatása, adatok átadása, számla parkolása.

\section{Fejezet Összegzése}
\label{sec:prototype_summary}
% A fejezet összefoglalása, átvezetés az eredmények értékelésére.s