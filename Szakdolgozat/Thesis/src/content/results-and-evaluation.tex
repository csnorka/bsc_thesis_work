\chapter{A Prototípus Tesztelése és Eredmények Értékelése}
\label{chap:ertekeles}

\section{Bevezetés: A Tesztelés Módszertana}
\label{sec:eval_intro}
% A tesztelés célja és módszerei.
% Milyen környezetben történt a tesztelés (pl. BTP Trial).

\section{Tesztelési Esetek (Test Cases) Bemutatása}
\label{sec:eval_test_cases}
% Különböző teszt forgatókönyvek (pl. sikeres, sikertelen, emberi beavatkozást igénylő).

\subsection{1. Teszteset: Sikeres "Touchless" Feldolgozás}
\label{ssec:eval_test_touchless}
% Egy tökéletes számla feldolgozása magas konfidencia pontszámmal, automatikus jóváhagyással és parkolással.

\subsection{2. Teszteset: Emberi Beavatkozás (Alacsony Konfidencia)}
\label{ssec:eval_test_low_confidence}
% Egy rosszul szkennelt számla, ahol a DOX alacsony megbízhatósági pontszámot ad.
% A folyamat megáll, és a "My Inbox"-ba küld egy javító feladatot.

\subsection{3. Teszteset: Elutasítási Folyamat}
\label{ssec:eval_test_rejection}
% A jóváhagyó menedzser az űrlapon "Elutasítva" gombra kattint.
% A folyamat kezelése, értesítés küldése.

\section{Eredmények Összehasonlítása az "As-Is" Folyamattal}
\label{sec:eval_comparison}
% A manuális (\ref{ssec:valos_vilag}) vs. automatizált folyamat (\ref{chap:megvalositas}) összevetése.

\subsection{Feldolgozási Idő (Átfutási Idő) Csökkenése}
\label{ssec:eval_time_reduction}
% Pl. Manuális: 14 nap -> Automatikus: 1 nap.

\subsection{Hibakockázatok Csökkenése}
\label{ssec:eval_error_reduction}
% Az adatrögzítési hibák, duplikált számlák kiküszöbölése.

\subsection{Átláthatóság Növekedése}
\label{ssec:eval_transparency}
% Az e-mailes "fekete lyuk" felszámolása.
% A Work Zone és a BPA Monitor felületének előnyei.

\section{Továbbfejlesztési Lehetőségek}
\label{sec:eval_future_work}
% Jövőbeli fejlesztési irányok (pl. Process Mining integráció, teljes könyvelés automatizálása).

\section{Fejezet Összegzése}
\label{sec:eval_summary}
% A fejezet összefoglalása, átvezetés a Konklúzió fejezetre (\ref{chap:osszefoglalas}).