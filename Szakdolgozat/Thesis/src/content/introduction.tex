
\chapter{\bevezetes}
\label{chap:bevezetes}


\section{Témafelvezetés: A Pénzügyi Funkció Stratégiai Átalakulása}
\label{sec:temafelvetes}

A pénzügyi funkció (\textit{Finance}) drámai átalakuláson megy keresztül, amelyet a folyamatos költségcsökkentési nyomás és a mélyebb üzleti betekintést igénylő stratégiai tanácsadói szerep iránti növekvő igény egyaránt vezérel. A hagyományos, reaktív és tranzakció-fókuszú ,,eredménykimutató'' (\textit{scorekeeper}) szerepkörből a pénzügy egyre inkább proaktív, ,,stratégiai értékteremtővé'' (\textit{strategic value creator}) válik.

\subsection{A Szerepkör Forradalma: A Tranzakcióktól a Stratégiáig}
\label{ssec:szerepkor_forradalma}

A pénzügy jövőbeli alapvető szerepváltása a tranzakciók feldolgozásától a stratégiai partnerség felé mozdul el. Ezt a trendet a piacvezető tanácsadó cégek egyértelműen alátámasztják. A Deloitte kiemeli, hogy a pénzügy szerepe ,,a kontrollálástól a stratégiai tervezésig'' (angolul: \textit{from controlling to strategic planning}) mozdul el, és egy ,,előretekintő üzleti funkcióvá válik, amely alakítja az üzleti irányt, ahelyett, hogy egyszerűen a múltbeli eredményekről számolna be.''

Ahogy a PwC elemzése fogalmaz, a ,,pénzügy a pénzügyért'' (\textit{finance for finance}) -- vagyis a hagyományos funkciók hatékonyságának növelése -- továbbra is fontos, de a jövő kulcsa a ,,pénzügy az üzletért'' (\textit{finance for business}), amely mélyebb betekintést nyújt az egész szervezetben.

\subsection{A Változás Gátja: A Manuális Munka Fogságában}
\label{ssec:valtozas_gatja}

Ez az evolúció azonban nem könnyű. A stratégiai partnerség legnagyobb akadálya, hogy a magasan képzett pénzügyi csapatok idejét felemésztik a manuális, ismétlődő feladatok.

A Deloitte nyíltan kimondja, hogy ,,Sok pénzügyi csapat még mindig manuális folyamatokra támaszkodik... ami lassítja a döntéshozatalt, és korlátozza a pénzügyi vezető (CFO) képességét, hogy teljes mértékben stratégiai üzleti partnerként lépjen fel.'' Egy sokat idézett McKinsey tanulmány számszerűsíti a problémát: becslésük szerint a pénzügyi tevékenységek 42\%-a már ma is teljesen automatizálható a meglévő technológiákkal, további 19\% pedig nagyrészt az lenne.

Ezen akadály leküzdése miatt válik a technológia az átalakulás abszolút motorjává.

\subsection{A Technológia Mint Motor: Automatizáció, AI és Adatok}
\label{ssec:technologia_motor}

Ez a szerepváltás elképzelhetetlen a technológia és az adatok együttes forradalma nélkül.

\paragraph{A ,,Finance Factory'' és az Automatizáció:} A fókusz a Deloitte által ,,pénzügyi gyárnak'' (\textit{finance factory}) nevezett koncepcióban az operatív, back-office feladatokról egyértelműen a pénzügyi betekintést nyújtó front-office felé tolódik. A tranzakcionális tevékenységek szinte teljes automatizálása várható, ami a Gartner szerint ,,a tranzakciós testreszabás végéhez'' (\textit{the end of transactional customization}) vezet. A PwC szövege ezt úgy írja le, mint a ,,háromszög megfordítása'': a technológia (AI, ML) lehetővé teszi, hogy a korábban tranzakciókkal terhelt széles bázis helyett a fókusz a cselekvésre ösztönző üzleti intelligenciára kerüljön.

\paragraph{A Példa: Számlafeldolgozás (AP):} Ez a stratégiai törekvés már a legalapvetőbb ,,back-office'' funkciókban is megjelenik. Az Ardent Partners jelentése szerint még a Számlafeldolgozási (Accounts Payable) osztályok is ,,helyet követelhetnek a stratégiai asztalnál'', mivel az automatizálás révén felszabadult idejüket már nem adatpötyögéssel, hanem készpénzmenedzsmenttel és szállítói kapcsolatok elemzésével tölthetik.

\paragraph{Mesterséges Intelligencia (AI) és Analitika:} A jövő már nem csak az egyszerű automatizálásról szól. A Gartner ,,AI ügynökökből álló munkaerőt'' (\textit{a workforce of AI agents}) és ,,gép-dominálta döntéshozatalt'' (\textit{machine-dominated decision making}) vizionál. A pénzügyi tervezési és elemzési (FP\&A) funkció átalakításának középpontjában a prediktív analitika és az AI-eszközök állnak.

\paragraph{Az Adat Mint Alapfeltétel:} Az összes jelentés egyetért abban, hogy a jövő pénzügyi funkciójának alapja a jó minőségű adat. Ahogy a Deloitte fogalmaz: a technológia nem lesz ,,csodaszer'' a megfelelő adatarchitektúra nélkül. A PwC kiemeli, hogy a vállalati adatmodellekbe és digitális képességekbe történő dedikált befektetés hiányában a pénzügyi vezetők nehezen tudnak majd valódi értéket teremteni.

\subsection{Új Fókuszterületek: Tőkeáramlás és Szabályozói Megfelelés}
\label{ssec:uj_fokuszteruletek}

A stratégiai tanácsadói szerepkör két új, kiemelt fókuszterületet hoz előtérbe, amelyeket a PwC elemzése részletesen tárgyal:

\paragraph{Tőke és Cash Flow (\textit{Capital and Cash Flow}):} A bizonytalan gazdasági környezet miatt a készpénzbeszedésre és a működő tőkére irányuló figyelem megnövekedett. Kulcsfontosságúvá válik a ,,megrendeléstől a készpénzbeérkezésig'' (\textit{order-to-cash}) folyamat automatizálása. Mivel a cégfelvásárlások és egyéb tranzakciók (M\&A) 2025-ben várhatóan felpörögnek, a CFO-knak újra kell értékelniük, hogy képesek-e következetesen készpénzre váltani a nyereséget.

\paragraph{Szabályozói Jelentések (\textit{Regulatory Reporting}):} Az új, komplex szabályozási követelmények (mint az OECD Pillar Two, ESG) drámaian megnövelik az átláthatóság és a részletes (granuláris) adatszolgáltatás iránti igényt. A PwC felmérése szerint a pénzügyi vezetők 70\%-a kockázatként tekint az amerikai szabályozói környezetre. Sok vállalat integrált megoldások híján manuális, szigetszerű adatgyűjtéssel pazarolja az erőforrásait.

\subsection{Az Átalakuló Működési Modell és a ,,Tehetség-Válság''}
\label{ssec:mukodesi_modell}

Az új feladatok új munkaszervezést és újfajta munkatársakat igényelnek.

\paragraph{A Tehetség Kérdése:} A szükséges készségek radikálisan megváltoznak. A hagyományos számviteli tudás helyett a jövő pénzügyi munkatársának erős kommunikációs készségekre, proaktivitásra és ,,digitális hozzáértésre'' lesz szüksége. A Deloitte ezt ,,tehetségekért folyó háborúnak'', míg a Gartner egyenesen ,,pénzügyi tehetség-válságnak'' (\textit{finance talent crash}) nevezi, rámutatva, hogy 2019 és 2021 között 300 000 könyvelő hagyta el a pályát. A PwC hangsúlyozza, hogy a képzett tehetségekért folyó verseny kemény, ezért a szervezetnek innovatív kultúrát kell kínálnia.

\paragraph{Új Működési Modellek:} A költségek kordában tartása mellett a szervezeteknek új erőforrásokat is be kell vonniuk. A Deloitte kiemeli, hogy a COVID-19 által tesztelt távmunka és hibrid modellek tartósan megmaradnak. Emellett a PwC elemzése rámutat, hogy sok szervezet harmadik fél (pl. menedzselt szolgáltatók) felé szervezi ki a nem alapvető tevékenységeket. Ennek célja már nem csupán a költségcsökkentés, hanem a szakemberhiány pótlása és a belső erőforrások felszabadítása az értéknövelő, stratégiai feladatok elvégzésére.

\section{Problémafelvetés: A Manuális Számlafeldolgozás Költségei}
\label{sec:problemafelvetes}

Ahogy a témafelvezetésben láthattuk, a pénzügyi osztály stratégiai szerepvállalása és a manuális számlafeldolgozás valósága között mély szakadék tátong. A szakdolgozat által megoldani kívánt probléma tehát a manuális számlafeldolgozásból eredő közvetlen és közvetett költségek halmaza, amely a legtöbb szervezetnél továbbra is erősen humán-intenzív feladat.

A közvetlen költségek a legnyilvánvalóbbak: a magasan képzett pénzügyi munkatársak idejüket azzal töltik, hogy PDF-ről vagy papírról adatokat gépelnek át a vállalati rendszerbe (ERP), mint például az SAP. Ez a ,,forgószék'' (\textit{swivel-chair}) probléma – ahol a munkatárs egyik monitorról a másikra (vagy papírról a monitorra) viszi át az adatokat – nemcsak lassú és drága, de rendkívül alacsony hozzáadott értékű. Ehhez társul a gyakran e-mailben vagy papíron keringtetett jóváhagyatási folyamat, amely lassú, átláthatatlan és nehezen követhető.

Ennél azonban súlyosabbak a folyamat rejtett, közvetett költségei, amelyek a stratégiai célokat is aláássák.

\paragraph{Pénzügyi veszteség:} A manuális adatrögzítés elkerülhetetlenül hibákhoz vezet, legyen szó elgépelt összegekről, rossz bankszámlaszámról, vagy az egyik leggyakoribb és legköltségesebb hibáról: a duplikált számlák kifizetéséről.

\paragraph{Határidők mulasztása:} A lassú, e-mail alapú jóváhagyatás miatt a vállalatok gyakran kicsúsznak a fizetési határidőkből, ami késedelmi kötbéreket vonhat maga után. Ennél is jelentősebb pénzügyi veszteséget okoz az elvesztett skontó (\textit{early payment discount}) lehetősége: a vállalat nem tud élni a korai fizetésért cserébe felajánlott árengedménnyel, mert a számla még ,,valahol a rendszerben'' várakozik jóváhagyásra.

Végül, a manuális folyamat legnagyobb stratégiai hátránya az átláthatóság teljes hiánya. A számlák egy ,,fekete lyukba'' kerülnek (pl. egy közös e-mail postafiókba), és a menedzsmentnek nincs valós idejű rálátása arra, hogy mely számla hol tart a feldolgozásban, ki az aktuális felelős, és ami a legfontosabb: mekkora a vállalat pontos, aktuális kötelezettségállománya (\textit{liability}). Ez az átláthatóság hiánya lehetetlenné teszi a hatékony készpénzmenedzsmentet és aláássa a pénzügyi osztály azon képességét, hogy a Bevezetésben felvázolt stratégiai partneri szerepet betöltse.

\section{A Megoldás Iránya: Hyperautomation és az SAP Build Platform}
\label{sec:megoldas_iranya}

Az 1.2. pontban vázolt komplex problémákra -- az adatrögzítéstől a jóváhagyási láncon át az átláthatóság hiányáig -- a válasz már nem egyetlen, izolált technológia. Míg az automatizálási hullám korai szakaszát a \textit{Robotic Process Automation} (RPA) uralta, amely elsősorban az ismétlődő, szabályalapú kattintásokat váltotta ki, a piac felismerte, hogy ez önmagában kevés. A valódi, végponttól végpontig tartó megoldás a \textit{Hyperautomation} (hiperautomatizálás) koncepciója. Ez nem egyetlen eszköz, hanem egy üzleti stratégia, amely több technológia -- köztük a Mesterséges Intelligencia (AI), a folyamatmenedzsment (BPM) és az RPA -- integrált alkalmazását jelenti a folyamatok újratervezésére.

A világ egyik vezető vállalatirányítási szoftvercége, az SAP, felismerte ezt az igényt. Válaszuk az SAP Business Technology Platform (BTP), egy egységes, felhőalapú platform, amely az adatok, az analitika, az integráció és az automatizálás eszközeit fogja össze. Az SAP stratégiájának kulcsa a \textit{Low-Code/No-Code} (LCNC) filozófia bevezetése. Ennek célja, hogy az automatizálást és az alkalmazásfejlesztést ,,kivigye'' a professzionális IT-fejlesztők kezéből, és közelebb hozza az üzleti felhasználókhoz, az úgynevezett ,,citizen developer''-ekhez (\textit{amatőr fejlesztőkhöz}), akik a folyamatokat a legjobban ismerik.

Ennek a hiperautomatizálási stratégiának a BTP platformon belüli zászlóshajója, és egyben e szakdolgozat központi vizsgált eszköze, az SAP Build Process Automation (BPA). A BPA ereje abban rejlik, hogy egyetlen, vizuális felületen egyesíti a hiperautomatizáláshoz szükséges három kulcsképességet, amelyek pontosan lefedik a számlafeldolgozás problémáit:
\begin{itemize}
    \item \textbf{Mesterséges Intelligencia (AI):} Strukturálatlan adatok, például PDF vagy szkennelt számlák tartalmának automatikus kinyerése (\textit{Document Information Extraction}).
    \item \textbf{Folyamatmenedzsment (BPM/Workflow):} Összetett, emberi beavatkozást igénylő jóváhagyási és ellenőrzési láncok grafikus modellezése és futtatása.
    \item \textbf{Robotika (RPA):} Automatizált botok futtatása, amelyek adatokat írhatnak be vagy olvashatnak ki más, akár régebbi rendszerekből (pl. maga az SAP S/4HANA).
\end{itemize}

\section{A Szakdolgozat Célkitűzései és Kutatási Kérdései}
\label{sec:celkituzesek}

Az előzőekben felvázolt elméleti háttér és a beazonosított üzleti probléma (a manuális számlafeldolgozás költségei) alapján a jelen szakdolgozat a következő fő célkitűzést fogalmazza meg:

\begin{quote}
\textbf{A szakdolgozat fő célja, hogy a manuális számlafeldolgozás fent vázolt problémáira választ adva, megtervezzen és megvalósítson egy automatizált számlakezelési prototípust az SAP Build Process Automation felhőalapú környezetében.}
\end{quote}

Ezen átfogó cél elérése érdekében a dolgozat a következő konkrét részcélokat, egyben kutatási kérdéseket tűzi ki:

\begin{description}
    \item[Elméleti célkitűzés:] A dolgozat célja részletesen bemutatni az SAP Build Process Automation (BPA) technológiai eszköztárát és az azt körülölelő hiperautomatizálási, valamint low-code/no-code (LCNC) koncepciókat.
    \item[Kutatási kérdés:] Milyen komponensekből áll az SAP BPA, és hogyan kapcsolódnak ezek a hiperautomatizálás modern trendjeihez?

    \item[Elemzési célkitűzés:] A dolgozat célja elemezni egy tipikus vállalati számlafeldolgozási gyakorlatot, azonosítani annak manuális lépéseit, főbb hibalehetőségeit és szűk keresztmetszeteit (\textit{bottlenecks}).
    \item[Kutatási kérdés:] Hol keletkeznek a legnagyobb költségek és a legtöbb hiba a manuális AP folyamat során, és melyek azok a pontok, amelyek automatizálással javíthatók?

    \item[Gyakorlati célkitűzés (Prototípus):] A dolgozat fő gyakorlati célja egy működőképes prototípus (\textit{Proof of Concept}) létrehozása. Ennek a megoldásnak képesnek kell lennie egy beérkező számladokumentum (pl. PDF) adatainak automatikus kinyerésére (AI), az adatok üzleti szabályok szerinti érvényesség-vizsgálatára, és egy előre definiált, többszintű jóváhagyási folyamat (\textit{workflow}) elindítására.
    \item[Kutatási kérdés:] Megvalósítható-e az SAP BPA eszközeivel egy olyan integrált prototípus, amely a számla beérkezésétől a jóváhagyásáig kezeli a folyamatot?

    \item[Értékelési célkitűzés:] Végül a dolgozat célja a megvalósított prototípus működésének tesztelése és hatékonyságának értékelése a korábban elemzett manuális folyamathoz képest.
    \item[Kutatási kérdés:] Mennyivel csökkenti a prototípus a feldolgozási időt és a manuális hibák lehetőségét a hagyományos, e-mail és kézi adatrögzítés alapú módszerhez viszonyítva?
\end{description}

\section{A Vizsgálat Fókusza és Korlátai}
\label{sec:fokusz_korlatok}

A szakdolgozat célkitűzéseinek (1.4. pont) reális teljesíthetősége érdekében elengedhetetlen a vizsgálat fókuszának és korlátainak egyértelmű meghatározása.

Fontos hangsúlyozni, hogy a dolgozat gyakorlati része egy prototípus (\textit{Proof of Concept}, PoC) elkészítésére vállalkozik. A cél nem egy teljeskörű, éles vállalati bevezetésre kész megoldás létrehozása, hanem annak koncepcionális bizonyítása, hogy az SAP Build Process Automation technológia alkalmas a beazonosított üzleti probléma (manuális számlafeldolgozás) hatékony kezelésére.

A prototípus a teljes \textit{Purchase-to-Pay} (P2P) folyamat egy szűk, de kritikusan fontos szakaszára fókuszál: a számla beérkezésétől a könyvelésre való előkészítéséig. A folyamat definiált végpontja a számla ,,parkolása'' (\textit{invoice parking}) az SAP rendszerben. Ez azt jelenti, hogy a számla adatstrukturálása, üzleti szabályok szerinti ellenőrzése és a felelősök általi jóváhagyása megtörtént, és a bizonylat előkészített státuszba került.

A dolgozat nem terjed ki a tényleges főkönyvi könyvelés (\textit{invoice posting}) automatizálására, sem pedig a pénzügyi teljesítés, azaz a kifizetési futtatás (\textit{payment run}) lépéseire. Továbbá a megvalósítás az SAP Business Technology Platform (BTP) ingyenes próbaverziós (\textit{Trial}) vagy fejlesztői környezetében történik, ami befolyásolhatja az elérhető funkciók körét és a rendszer teljesítményét egy éles vállalati környezethez képest.

\section{A Dolgozat Felépítése}
\label{sec:dolgozat_felepitese}

A szakdolgozat a Bevezetésben lefektetett célkitűzések elérése érdekében a következő logikai szerkezet szerint épül fel:

A \textbf{Második Fejezet} bemutatja a dolgozat elméleti alapjait és a felhasznált technológiát. Részletesen tárgyalja a hiperautomatizálás koncepcióját, valamint az SAP Build Process Automation felhőalapú platform eszközkészletét, kitérve annak AI, workflow és RPA képességeire.

A \textbf{Harmadik Fejezet} a vizsgált üzleti folyamatot, a szállítói számlák kezelését elemzi. Először bemutatja a standard SAP ,,tankönyvi'' folyamatát, majd azt összeveti egy tipikus vállalati gyakorlattal, feltárva annak manuális hibáit, költségeit és szűk keresztmetszeteit. Ez a fejezet alapozza meg a prototípus szükségességét.

A \textbf{Negyedik Fejezet} a dolgozat gyakorlati magja, amely részletesen bemutatja az automatizált prototípus tervezésének és megvalósításának lépéseit. Ez magában foglalja az adatkinyerés (AI) konfigurálását, a jóváhagyási logika (workflow) felépítését és az SAP rendszerrel való integrációt (a számla parkolását).

Az \textbf{Ötödik Fejezet} a prototípus tesztelésének eredményeit és a megoldás értékelését tartalmazza. A fejezet összehasonlítja az automatizált megoldás hatékonyságát (pl. feldolgozási idő, hibalehetőségek) a Harmadik Fejezetben azonosított manuális folyamattal, és javaslatokat fogalmaz meg a lehetséges továbbfejlesztésre.

Végül a \textbf{Hatodik Fejezet} összegzi a dolgozat kutatási eredményeit, válaszol a Bevezetésben feltett kutatási kérdésekre, és levonja a végső következtetéseket.