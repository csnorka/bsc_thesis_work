\chapter{Standard és Vállalati Számlafeldolgozási Folyamat Elemzése}
\label{ch:process-analysis}

\section{Bevezetés: Az ,,As-Is'' Folyamat Megértése}
\label{sec:as-is-introduction}

Ennek a fejezetnek a célja, hogy részletesen elemezze a szállítói számlafeldolgozás aktuális üzleti folyamatát, az úgynevezett ,,As-Is'' állapotot.
A probléma mélyebb megértése érdekében az elemzést két, egymásra épülő szintre bontjuk.
Elsőként bemutatjuk a standard SAP folyamatot, vagyis azt a ,,tankönyvi modellt'', ahogyan a számlakezelés egy ideálisan konfigurált vállalatirányítási rendszerben működik.
Ezt követően összevetjük az ideális állapotot a tipikus vállalati gyakorlattal: azzal a valósággal, ahogyan a folyamat a legtöbb szervezetnél zajlik, ahol a rendszer beépített képességeit gyakran manuális, e-mail alapú ,,foltokkal'' és ad hoc megoldásokkal egészítik ki.
A két modell – az elméleti és a valós – közötti eltérés, azaz a ,,rés'' fogja pontosan kijelölni azokat a nehézségeket, költséges szűk keresztmetszeteket és hibaforrásokat, amelyekre a \textbf{\ref{chap:megvalositas}}.
fejezetben tervezett és megvalósított SAP Build Process Automation prototípus célzott megoldást kínál.
\subsection{A Standard SAP Számlafeldolgozási Folyamat (,,A Tankönyvi Modell'')}

A szállítói számlák kezelésének megértéséhez és értékeléséhez elengedhetetlen annak az ideális, elméleti modellnek a bemutatása, amelyet az SAP rendszer a maga teljes körűen konfigurált és integrált állapotában kínál.
Ez az ún. ,,tankönyvi modell'' szolgál majd később etalonként, amellyel a valós vállalati gyakorlatot összevetjük.
Ezen elméleti keret megalkotása mögött az SAP FI (Pénzügy) és MM (Anyaggazdálkodás) moduljainak zökkenőmentes együttműködése húzódik meg, mely lehetővé teszi a beszerzési folyamat teljes körű lezárását – a megrendeléstől a fizetésig.
Ez az alfejezet ennek az integrált, standard folyamatnak a főbb elveit, dokumentumait és tranzakcióit mutatja be, kiemelve azokat a beépített automatizálási és ellenőrzési pontokat, amelyek a rendszer hatékonyságának és megbízhatóságának alapját képezik.
\subsubsection{Az Alapelv: A Háromoldalú Egyeztetés (Three-Way Match)}

A logisztikai számlaellenőrzés (LIV) és a teljes Procurement-to-Pay (P2P) folyamat központi alapelve a háromoldalú egyeztetés (Three-Way Match) \cite{SAP-LIV, RAMP-Invoice-Verification}.
Ez a könyvelési ellenőrzési mechanizmus nem csupán technikai funkció, hanem alapvető pénzügyi biztonsági háló, melynek egyetlen célja van: biztosítani, hogy a vállalat kizárólag azért fizessen, amit előzetesen megrendelt és amit ténylegesen meg is kapott.
A folyamat az alábbi három alapvető dokumentum egyidejű, automatikus összevetésén alapul, melyek együttesen lefedik a beszerzés ezen üzleti életciklusát:

\begin{itemize}
\item \textbf{Beszerzési Rendelés (Purchase Order - PO) - ,,A megállapodás'':} Ez a dokumentum (létrehozva az ME21N tranzakcióval) tartalmazza a vállalat szándékát és a szállítóval kötött formális megállapodást: mit rendeltünk, milyen áron és milyen feltételek mellett?
A PO határozza meg az elvárásokat.

\item \textbf{Anyagbevételezés (Goods Receipt - GR) - ,,A fizikai átvétel'':} Ez a bizonylat (rögzítve a MIGO tranzakcióval) igazolja a tényleges teljesítést: mit kaptunk meg a raktárban valójában?
A GR nemcsak a készletnövekedést okozza, hanem egyben a szállítói kötelezettség előkészítését is jelenti a GR/IR (Goods Receipt/Invoice Receipt) számlán.
\item \textbf{Számlabefogadás (Invoice Receipt - IR) - ,,A pénzügyi követelés'':} A szállító által küldött számla (feldolgozva a MIRO tranzakcióban) tartalmazza a fizetési igényt: miről és mekkora összegben kér fizetést a szállító?
\end{itemize}

A háromoldalú egyeztetés folyamatának logikája a MIRO tranzakcióban materializálódik. Amikor a pénzügyi ügyintéző egy PO-hoz hivatkozó számlát rögzít, az SAP rendszer automatikusan és szorongatottan összeveti a számla minden egyes tételének mennyiségét és árát a hivatkozott Beszerzési Rendelés (PO) és az arra történt Anyagbevételezés (GR) adataival.
Ez a ,,három szintű'' egyeztetés (Three Levels) gyakorlatilag a GR/IR folyamat magja.
Az egyezés hiánya esetén a rendszer a konfigurált toleranciahatárok alapján akár automatikusan blokkolhatja a számlát fizetésre, amelyet csak a felelős beszerző (Buyer) oldhat fel (például az MRBR jelentés segítségével), ezzel biztosítva a problémák azonosítását és megoldását.
A folyamat stratégiai jelentőségét az adja, hogy hatékonyan megakadályozza a gyakori operatív kockázatokat, mint például a duplikált vagy hibás számlák, a nem rendelés alapján történt szállítások, vagy a nem átvett áruk kifizetése \cite{SAP-LIV}.
A háromoldalú egyeztetés tehát nem csupán egy technikai lépés, hanem a vállalat pénzügyi integritásának és a beszerzési folyamatok épségének záloga.
\subsubsection{A Folyamat Két Fő Típusa az SAP-ban}

Az SAP rendszerében a szállítói számlák feldolgozásának módját alapvetően az határozza meg, hogy a számla kapcsolódik-e egy létező beszerzési rendeléshez (Purchase Order - PO) \cite{SAP-Managing-Invoices, SAP-Non-PO-Invoices}.
Ez a megkülönböztetés meghatározza a használandó tranzakciót, az ellenőrzési folyamatot és a feldolgozás automatizáltsági fokát.
A két alapvető típus a PO-alapú (logisztikai) és a nem PO-alapú (pénzügyi) számlarögzítés.

\paragraph{1.
PO-alapú (Logisztikai) Számlarögzítés (MIRO tranzakció) \cite{SAP-Managing-Invoices}}

Ez a folyamat a Logisztikai Számlaellenőrzés (LIV) magja, és az anyagok, szolgáltatások szabályozott beszerzésére jellemző.
\textbf{Alkalmazási kör:} Akkor használatos, amikor a számla egy létező és érvényes beszerzési rendeléshez (PO) kapcsolódik.
Ez jellemzően alapanyagok, késztermékek, alkatrészek beszerzésére, valamint olyan szolgáltatásokra vonatkozik, amelyeket előre leegyeztettek és PO-n rögzítettek (pl. külső szakértői munka, karbantartási szerződés).
\textbf{Feldolgozás lépései:}
\begin{enumerate}
\item Az ügyintéző a MIRO tranzakcióban megadja a beszerzési rendelés számát.
\item A rendszer azonnal automatikusan feltölti a számla tételeit a PO és a hozzá tartozó áruátvételi (GR) adatok alapján.
Megjelenik a rendelés összes nyitott tétele, a már átvett mennyiségek és a megállapodott árak.
\item Az ügyintéző feladata lecsökken a fejadatok (számlaszám, dátum, fizetési feltételek) és a lehetséges eltérések (pl. minimális mennyiségi vagy árkülönbözet) kezelésére.
A rendszer a ,,háromoldalú egyeztetés'' elvével automatikusan összeveti a számla, a PO és a GR adatait.
\end{enumerate}

\textbf{Eredmény és kontroll:} A folyamat végeredménye egy automatikusan ellenőrzött és könyvelhető számla.
A rendszer kiszámolja a különbözetet (ha van), és ha az a konfigurált toleranciahatárokon belül van, a számla fizetésre jóváhagyott státuszt kap.
Ha az eltérés túl nagy, a rendszer fizetési blokkot állít be, amelyet csak a felelős beszerző v. vezető oldhat fel.
Ez a folyamat biztosítja, hogy a vállalat kizárólag a megrendelt és átvett árukért vagy szolgáltatásokért fizet.

\paragraph{2.
Nem PO-alapú (Pénzügyi) Számlarögzítés (FB60 tranzakció) \cite{SAP-Non-PO-Invoices}}

Ezt a folyamatot olyan kiadásokra használják, amelyeket nem előzték meg formális beszerzési eljárással és PO-kiállítással.
\textbf{Alkalmazási kör:} Akkor kerül sor erre, amikor a számla nem köthető egyetlen beszerzési rendeléshez sem.
Ilyen tipikusan a közüzemi számlák (villany, víz, gáz), bérleti díjak, általános szolgáltatások (pl. jogi tanácsadás, reklámkampány) vagy egyéb, előre nem tervezett üzleti kiadások.
\textbf{Feldolgozás lépései:}
\begin{enumerate}
\item Az ügyintéző az FB60 tranzakcióban közvetlenül rögzíti a számla adatait.
\item Mivel nincs PO vagy GR, amire a rendszer hivatkozhatna, az automatikus adatfelület és egyeztetés hiányzik.
\item \textbf{Kritikus manuális lépés:} Az ügyintézőnek teljes mértékben manuálisan kell elvégeznie a kontírozást.
Ez azt jelenti, hogy minden egyes számlatételhez ki kell választania a megfelelő főkönyvi számlát (amely meghatározza a költségnemet, pl. ,,Villamos energia költsége'') és a költséghelyet (amely meghatározza, hogy melyik osztály vagy projekt viseli a költséget, pl. ,,Épületüzemeltetés'' vagy ,,Marketing Osztály'').
\end{enumerate}

\textbf{Eredmény és kockázat:} A folyamat egy közvetlenül a főkönyvbe könyvelt számlával zárul.
A feldolgozás jelentősen munkaigényesebb és hibákra érzékenyebb, mivel nem történik automatikus egyeztetés.
A pontosság teljes mértékben az ügyintéző szakértelmén és figyelmességén múlik.
Egy helytelen költséghely vagy főkönyvi számla megadása torzított költségelszámoláshoz és helytelen pénzügyi kimutatásokhoz vezet.
Emiatt ezek a számlák gyakran több lépcsős belső jóváhagyási folyamaton esnek át, mielőtt kifizetésre kerülnének.
\begin{table}
\centering
\caption{Összefoglaló táblázat a PO-alapú és nem PO-alapú számlafeldolgozás összehasonlításához}
\label{tab:invoice-processing-comparison}
\begin{tabular}{|p{0.2\textwidth}|p{0.35\textwidth}|p{0.35\textwidth}|}
\hline
\textbf{Szempont} & \textbf{PO-alapú (MIRO)} & \textbf{Nem PO-alapú (FB60)} \\
\hline
Tranzakció & \texttt{MIRO} & \texttt{FB60} \\
\hline
Elv & Háromoldalú egyeztetés (PO-GR-Számla) & Közvetlen pénzügyi könyvelés \\
\hline
Automatizáció & Magas (rendszer javasol adatokat) & Alacsony (manuális kontírozás) \\
\hline
Kontroll & Erős (rendszer blokkol eltéréseket) & A folyamat és a jóváhagyás feladata \\
\hline
Kockázat & Alacsony & Magas (emberi hiba) \\
\hline
Példa & Alumínium öntvény beszerzése & Villanyszámla, Hirdetési költség \\
\hline
\end{tabular}
\end{table}

\subsection{Standard Automatizálási és Manuális Vezérlési Pontok}

Az SAP rendszer kifinomultságát az is jelzi, hogy alapkiépítésében is tartalmaz jól meghatározott automatizálási mechanizmusokat és tudatosan kialakított manuális beavatkozási pontokat.
Ez a kettősség biztosítja a folyamatok hatékonyságát anélkül, hogy veszélyeztetné a pénzügyi kontrollok és a döntéshozatali lehetőség integritását.
Az automatizmusok a rutin feladatok gyors és hibatmentes elvégzését szolgálják, míg a manuális ,,fékek'' lehetővé teszik a szükséges szakértői felülvizsgálatot kivételes helyzetekben.
\subsubsection{Beépített Automatizációs Lehetőségek} \label{sssec:beepitett_auto}

Az SAP rendszer a számlafeldolgozás hatékonyságának és biztonságának növelése érdekében számos beépített automatizálási mechanizmust kínál, amelyek lehetővé teszik a manuális ellenőrzések mértékének csökkentését, miközben megtartják a pénzügyi kontrollok szilárd keretét.
Ezek az automatizmusok a folyamatok gyorsításán túl a humán hibák kiküszöbölésére és a költségek csökkentésére is összpontosítanak.
\paragraph{Toleranciahatárok (Tolerances) \cite{SAP-Tolerance-Keys-Part1, SAP-Tolerance-Keys-Part2}}

A toleranciahatárok az SAP egyik legkifinomultabb kontrollmechanizmusa, amely kétjegyű kódok (toleranciakulcsok) formájában valósul meg, és amelyek a rendszer által tárolt előre meghatározott korlátok alapján automatikusan blokkolják a számlákat fizetésre, amennyiben azok ára, mennyisége vagy egyéb paraméterei túllépik a megengedett határértékeket.
A mechanizmus nem csupán védelmet nyújt a túlfizetések ellen, hanem jelentős mértékben lecsökkenti a pénzügyi munkatársak terhelését is azáltal, hogy kiküszöböli a minden egyes számla egyedi manuális ellenőrzésének szükségességét.
A konfiguráció a T169G táblában történik, és a különböző üzleti helyzetekre szabottan több tucat speciális toleranciakulcs áll rendelkezésre.
A legfontosabb toleranciakulcsok közé tartozik a PP (Price Variance), amely a beszerzési rendelésen megállapított és a számlán szereplő egységár közötti eltérést ellenőrzi, vagy a DQ (Quantity Variance), amely a ténylegesen átvett és a számlázott mennyiség közötti különbséget monitorozza.
Azonban a rendszer ennél jóval részletesebb kontrollokat is biztosít: a BD (Small Differences) kulcs például automatikusan elszámolja a minimális összegű eltéréseket egy előre meghatározott ,,kis különbözetek'' főkönyvi számlára, miközben a KW (Variance from Condition Value) specifikusan a szállítási és egyéb mellék költségek egyeztetésére specializálódott.
A folyamat teljességét az MRBR tranzakció biztosítja, amely egy központosított felületet kínál az automatikusan blokkolt számlák áttekintéséhez, okának feltárásához és – adott esetben – manuális feloldásához.
\paragraph{ERS (Evaluated Receipt Settlement) \cite{SAP-ERS}}

Az Evaluated Receipt Settlement (ERS) a számlafeldolgozási automatizáció csúcspontját jelentő, radikálisan innovatív megközelítés, amely alapvetően újradefiniálja a hagyományos számlázási folyamatot.
Az ERS lényege, hogy a szállító és a vevő közötti előzetes megállapodás alapján a szállító egyáltalán nem küld fizikai vagy elektronikus számlát.
Ehelyett az SAP rendszer kizárólag a beszerzési rendelés (PO) és a tényleges áruátvétel (MIGO) adatai alapján automatikusan generálja és könyveli a számlabizonylatot a megállapodott árak és feltételek figyelembevételével.
Ennek a módszernek az alkalmazása számos stratégiai előnnyel jár: a beszerzési ciklus ideje drámaian lecsökken, hiszen a számla kézbesítésére és feldolgozására várási idő teljesen megszűnik.
Emellett a rendszer kiküszöböli a kézi adatrögzítésből adódó emberi hibákat, és megelőzi a mennyiségi vagy értékbeli eltérések kialakulását, mivel a számla mindig a PO és a GR pontos adatai alapján készül.
Az ERS különösen hatékonyan alkalmazható olyan szabványosított termékek és rutinszerű szolgáltatások esetében, ahol a szállítóival megbízható partneri kapcsolatot ápol, és a díjszabás stabil.
Fontos megjegyezni, hogy az ERS nem alkalmazható olyan esetekben, ahol a rendszeresítés tervezett vagy nem tervezett szállítási költségeket tartalmaz.
\paragraph{EDI/IDoc (Electronic Data Interchange / Intermediate Document) \cite{SAP-EDI}}

Az elektronikus adatcsere (EDI) az üzleti kommunikáció modern standardja, amely strukturált formátumban történő, rendszer és rendszer közötti adatcserét tesz lehetővé, ezzel teljes mértékben kiküszöbölve a papír alapú vagy emberi beavatkozással járó adatátvitelt.
Az SAP rendszerben ezt a folyamatot az úgynevezett Intermediate Document (IDoc) technológia valósítja meg, amely speciális üzenetformátumként szolgál a számlaadatok továbbítására.
Amikor egy szállító EDI-n keresztül küld számlát, az azt reprezentáló IDoc üzenet közvetlenül az SAP rendszerbe érkezik, ahol a rendszer automatikusan megkísérli a számla könyvelését a beszerzési rendelés előzményei alapján meghatározott javasolt mennyiség és érték figyelembevételével.
A folyamat során a rendszer a Customizing beállításainak megfelelően kezeli az esetleges eltéréseket: fizetési blokkolással, parkolással (invoice parking) vagy akár a rendszer által javasolt értékek használatával is történhet a könyvelés.
Az EDI/IDoc integráció nemcsak a feldolgozási sebességet növeli meg és csökkenti a adminisztratív terheket, hanem az adatpontosságot is maximalizálja, miközben teljes körű audit nyomvonalat biztosít a bejövő dokumentumokról.
\subsubsection{Tudatos Manuális Beavatkozási Pontok}

Míg az SAP rendszer kiterjedt automatizálási lehetőségeket kínál, a gyakorlatban számos olyan stratégiai pont van, ahol a tudatos manuális beavatkozás elkerülhetetlen és egyben értéket teremtő.
Ezek a pontok nem a rendszer hiányosságait jelzik, hanem éppen ellenőleg, olyan tervezett ,,fékeket'' és ellenőrzési mechanizmusokat testesítenek meg, amelyek lehetővé teszik a szükséges szakértői értékelést és döntéshozatalt a pénzügyi folyamatok kulcspontjain.
Ezek a manuális intervenciók biztosítják, hogy a rendszer rugalmassága megmaradjon a valós üzleti helyzetek kezelésében, miközben a pénzügyi kontrollok integritása sértetlen marad.
\paragraph{Manuális Számlarögzítés: Az Adatátvitel Kritikus Szűkkeresztmetszete}

A számlafeldolgozási lánc legjelentősebb manuális teherpróbája akkor következik be, amikor a számla nem elektronikus adatcsere (EDI) útján, hanem hagyományos papír alapon vagy strukturálatlan PDF formátumban érkezik.
Ebben az esetben az ügyintézőnek teljes mértékben manuálisan kell átvinnie a számla összes releváns adatát - számlaszám, dátum, szállítói adatok, tételek, árak, adók - az SAP rendszerbe a MIRO (beszerzési rendeléshez kapcsolódó) vagy FB60 (közvetlen könyvelésű) tranzakciók segítségével.
Ez a folyamat, amelyet gyakran ,,forgószék-adaptációként'' (,,swivel-chair integration'') ismernek, nem csupán időigényes és költséges, hanem kiemelten ki van téve az emberi hibáknak is.
Az elgépelések, számtranszpozíciók vagy hiányzó adatok komoly pénzügyi következményekkel járhatnak, beleértve a duplikált fizetéseket, a helytelen költségelszámolást vagy a meg nem érdemelt kötbérek kifizetését.
Jóllehet, ez a manuális lépés jelenleg még mindig elkerülhetetlen a legtöbb szervezetnél, és egyben a legfontosabb indoklást szolgáltatja a hiperautomatizálási megoldások, mint például az intelligens adatkinyerés (Document Information Extraction) bevezetésére.
\paragraph{Számlaparkolás: A Kontrollált ,,Félkész'' Állapot Művészete \cite{SAP-Parking, SAP-Holding-and-Parking, XSUITE-Parking-in-SAP}}

A számlaparkolás (MIR7 vagy FV60 tranzakciók) egy kifinomult köztes állapot, amely lehetővé teszi a számla adatainak biztonságos elmentését anélkül, hogy az végleges főkönyvi könyvelést generálna.
Ezt a funkcionalitást leginkább egy ,,fizetésre előkészített'' vagy ,,függőben lévő'' státuszként lehet leírni, ahol a dokumentum minden adata rendelkezésre áll, de a tényleges könyvelés és fizetés még nem történt meg.
A parkolás stratégiai jelentősége abban rejlik, hogy rugalmasságot biztosít a belső ellenőrzési folyamatok számára.
Egy számlát parkolni szokás, amikor az adatai bizonytalanok vagy hiányosak, további információkra van szükség a költséghelyes elosztáshoz, belső jóváhagyási körök függenek a véglegesítéstől, vagy egyszerűen csak a fizetési időzítést szeretnék optimalizálni.
A parkolt számla később bármikor elővehető, akár egy másik, magasabb szintű felhasználó (például egy költséghely-vezető vagy pénzügyi kontroller) által, aki átnézheti, kiegészítheti, korrigálhatja, és végül ,,élesre'' könyvelheti azt.
Ez a folyamat biztosítja, hogy csak a teljes körűen ellenőrzött és hivatalosan jóváhagyott számlák kerüljenek a vállalat főkönyvébe, ezzel erősítve a belső kontrollrendszer megbízhatóságát.
\paragraph{Blokkolt Számlák Feloldása: Az Eltérések Menedzsmentje \cite{JKE-Supply-Chain-Business-Process}}

Az SAP rendszer intelligens kontrollmechanizmusa, különösen a toleranciahatárok, gyakran automatikus fizetési blokkot alkalmaz azokon a számlákon, amelyek ára, mennyisége vagy egyéb paraméterei túllépik az előre meghatározott korlátokat.
Amikor egy ilyen blokk létrejön a MIRO tranzakció során, a rendszer ugyan létrehozza a számlabizonylatot, de azt egy speciális ,,fizetésre blokkolt'' státuszba helyezi, megakadályozva ezzel a pénztári futtatás (payment run) során történő kifizetését.
A blokk feloldása nem automatizált, hanem egy tudatos, felelősségteljes manuális lépés, amelyet általában a MRBR tranzakció (Feladáslistája a számláknak, blokk feloldással) segítségével hajtanak végre.
Ez a folyamat feltétlenül megköveteli, hogy egy előre meghatározott, illetékes felelős (legtöbbször a beszerző, aki ismeri a megrendelés hátterét, vagy egy pénzügyi vezető) személyesen felülvizsgálja az eltérés okát.
Ennek során értékeli, hogy az eltérés indokolt-e (például egy egyeztetett áremelkedés, szállítási problémából adódó többletköltség), dokumentálja a döntés indoklását, és végül manuálisan jóváhagyja a fizetést.
Ez a fék mechanizmus rendkívül értékes, mivel megakadályozza a jelentős vagy gyanús eltérésekkel rendelkező számlák automatikus kifizetését, és lehetővé teszi, hogy a megfelelő szakértői tudás bevonásával szolgáltassák ki a végső döntést.
Egyes toleranciakulcsok (például DQ vagy DW) esetén a blokk automatikusan is feloldódhat, ha a mögöttes probléma megszűnik (például a hiányzó áruátvétel rögzítésre kerül), de a legtöbb kritikusabb blokk (például PP áreltérés) kifejezetten emberi beavatkozást igényel a feloldásához.
\subsection{Egy Tipikus Vállalati Gyakorlat Elemzése (,,A Valós Világ'')} \label{ssec:valos_vilag}

Az SAP rendszer elméleti keretei és beépített automatizálási lehetőségei (\textbf{\ref{sssec:beepitett_auto}}) dacára a valós vállalati gyakorlat gyakran jelentős eltéréseket mutat az ideális ,,tankönyvi modelltől''.
Ebben a fejezetben egy fiktív, de a gyakorlatban jól megfigyelhető tipikus vállalati környezetet (,,As-Is'' állapot) elemzünk, amely bár technikailag rendelkezik SAP licenccel és alapvető konfigurációval, a fejlett automatizálási eszközök - mint például az SAP Build Process Automation - hiánya, valamint a történelmi teherként jelentkező elavult munkafolyamatok miatt nem képes kihasználni a rendszer teljes potencialitását.
Ez a diszkrepancia a technológia és a valós üzleti gyakorlat közötti szakadékot testesíti meg, amely számos operatív és stratégiai kihívást generál.
A vizsgált modell alapvető jellemzője a digitális és analóg folyamatok hibrid együttélése.
Míg a végső könyvelési lépések az SAP rendszerében történnek, addig a számlafeldolgozás előkészítő, ellenőrzési és jóváhagyási fázisai túlnyomórészt az SAP-on kívüli, manuális csatornákon keresztül zajlanak.
Ennek a hibrid modellnek a működését egy jól körülhatárolt, ismétlődő ciklus jellemzi, amely a következő kritikus lépésekből áll:

\begin{enumerate}
\item \textbf{Beérkezés: A Digitális Postafiók Kaotikus Világa} A folyamat kezdete egy dedikált, de gyakran túlterhelt e-mail címre történik (pl. szamlak@vallalat.hu), ahová a számlák mintegy 90\%-a PDF formátumban érkezik.
Ez a központi gyűjtőpont azonban gyakran átláthatatlan ,,digitális fekete lyukká'' válik, ahol a dokumentumok könnyen elveszhetnek a nagy mennyiségű beérkező levél között.
A probléma mélyebb gyökere a PDF formátum használatában rejlik - bár digitális, ez a formátum alapvetően strukturálatlan, és nem tartalmaz géppel olvasható adatokat, ami lehetetlenné teszi a közvetlen rendszerintegrációt.
Ahogyan Anichshuk (2025) is rámutat \cite{Kolleno-PDFs-to-EInvoices}, ,,a PDF egy statikus dokumentum, nem pedig strukturált adatokat tartalmazó fájl, amelyet a rendszerek képesek lennének olvasni'', ami elkerülhetetlenül manuális adatátvitelhez vezet.
\item \textbf{Manuális Szelektálás és Mentés: Az Adminisztratív Teher Első Szintje} Egy pénzügyi adminisztrátornak (AP ügyintéző) folyamatosan monitoroznia kell a postafiókot, manuálisan ki kell válogatnia a számlákat, letöltenie a PDF fájlokat, és azokat egy megosztott hálózati mappa előre meghatározott, de gyakran bonyolult mappaszerkezetében elhelyeznie.
Ez a látszólag egyszerű lépés rendkívül időigényes, és jelentős kockázati pontot jelent: az elveszett vagy rossz helyre mentett számlák hetekig, akár hónapokig is ,,eltűnhetnek'' a rendszerből, ami késedelmi kamatokhoz és megszakadt ellátási lánchoz vezethet.
A Parseur (2025) által közölt benchmark adatok szerint a manuális feldolgozás költsége 12-20 dollár számlánként terjed, ami ezen lépések összetettségét is jelzi \cite{Parseur-AI-Invoice-Processing}.
\item \textbf{,,Swivel-Chair'' Adatrögzítés: A Hibák Fő Forrása} A valódi operatív teher ekkor jelentkezik, amikor a pénzügyi ügyintéző párhuzamosan kell, hogy működjön az SAP felülete és a számla PDF-je között.
Az ügyintéző egyik monitoron megnyitja a letöltött számlát, a másikon pedig az MIRO (beszerzési rendeléshez kapcsolódó) vagy FB60 (közvetlen könyvelésű) tranzakciót, majd kézzel, manuálisan átviszi az összes szükséges adatot.
Ez a ,,forgószék-integráció'' (,,swivel-chair integration'') nemcsak rendkívül lassú (átlagosan 10-30 perc számlánként), de kiemelten ki van téve az emberi hibáknak is.
Az elgépelt számok, a rosszul másolt dátumok vagy a téves költséghely kijelölések komoly pénzügyi következményekkel járhatnak, beleértve a duplikált fizetéseket és a helytelen költségelszámolást.
\item \textbf{Jóváhagyási Kör: A Kommunikációs ,,Fekete Lyuk''} Amikor egy számla belső jóváhagyást igényel - legyen szó nem PO-alapú kiadásról (FB60) vagy a toleranciahatárok túllépéséről (MIRO) - a rendszer nem egy integrált workflow-ot indít el.
Ehelyett a felelős ügyintézőnek manuálisan kell parkolnia a számlát (FV60), majd e-mailben megkeresnie a megfelelő jóváhagyót (osztályvezetőt vagy beszerzőt), csatolva a parkolt bizonylat számát és a PDF számlát.
Ez a folyamat azonban gyakran összeomlik a valóságban: a vezetők, akik naponta több tucat hasonló kérést kapnak, nem tudják prioritizálni a számlajóváhagyásokat.
Ahogyan Kumar Mishra és munkatársai (2024) megállapították, a manuális intervenció igénybevétele jelentős időráfordítással jár, és a folyamat átláthatatlansága tovább rontja a helyzetet \cite{JKE-Supply-Chain-Business-Process}.
\item \textbf{Követés és Várakozás: Az ,,E-mail Üldözés'' Költői Köre} A vezetők elfoglaltsága és az e-mailek eltévedése miatt az AP ügyintézőnek napokkal, hetekkel később telefonon vagy újabb e-mailben kell ,,üldöznie'' (,,chasing'') a jóváhagyást.
Ez az úgynevezett ,,e-mail üldözéses'' folyamat nemcsak további adminisztratív terhet jelent, hanem komoly üzleti kockázatokhoz is vezethet.
A lassú átfutási idő miatt a vállalat gyakran nem tud élni a korai fizetési kedvezményekkel (skontó), ami jelentős pénzügyi veszteséget okoz, valamint késedelmi kamatok kifizetésére kényszerülhet.
A folyamat átláthatatlansága miatt a menedzsment nem rendelkezik valós idejű betekintéssel sem a folyamatban lévő számlák állapotába, sem a vállalat aktuális kötelezettségállományába.
\item \textbf{Manuális Véglegesítés: A Bizonytalan Lezárás} A folyamat végén a vezető laza válasza (,,OK'', ,,Rendben'') alapján - amely önmagában is audit kockázatot jelent - az AP ügyintéző megkeresi és megnyitja a parkolt bizonylatot, majd azt ,,élesre'' könyveli.
Ez a végső manuális lépés azonban további hibalehetőségeket hordoz: a rossz bizonylatszám megadása, a téves összeg beírása vagy egyszerűen csak az elavult információkon alapuló döntések további problémákhoz vezethetnek.
\end{enumerate}

\subsection{Problémafeltárás: A Manuális Gyakorlat Hibái és Költségei} \label{sec:problemfeltaras_hibai_koltsegei}

Ez a fejezet a szakdolgozat egyik legfontosabb elemzése, amely közvetlenül megindokolja a \textbf{\ref{chap:megvalositas}}.
fejezetben bemutatásra kerülő automatizált prototípus szükségességét. A manuális számlafeldolgozás nem csupán hatékonysági problémákat okoz, hanem jelentős pénzügyi kockázatokat és stratégiai akadályokat is jelent a vállalatok számára.
Az alábbiakban részletesen feltárjuk a hagyományos megközelítés minőségi és mennyiségi korlátait.
\subsubsection{Tipikus Manuális Hibák (Minőségi Kockázatok)}

A manuális folyamatok elkerülhetetlenül vezetnek olyan minőségi problémákhoz, amelyek közvetlen pénzügyi veszteségeket és működési kockázatokat eredményeznek.
\paragraph{Adatrögzítési hibák (Elgépelések)}

A ,,forgószék'' feladat, ahol az ügyintéző párhuzamosan dolgozik a PDF számla és az SAP felület között, kiemelten ki van téve az emberi hibáknak.
Az elgépelések - mint például 100.000 helyett 1.000.000 forint beírása, vagy dátumok felcserélése - nem csupán adminisztratív problémát jelentenek.
A Parseur (2025) kutatása szerint a manuálisan feldolgozott számláknál 1-2\%-os hibaaránya teljesen általános, ami nagy volumenű feldolgozás esetén komoly pénzügyi következményekkel jár \cite{Parseur-AI-Invoice-Processing}.
A ResolvePay (2025) által közölt adatok még aggasztóbb képet festenek: a számlák 39\%-a tartalmaz hibákat, és az AP szakemberek ezeknek csupán 39\%-át képesek észlelni az ellenőrzés során \cite{Resolvepay-17-Statistics-Showing}.
\paragraph{Duplikált Számlafeldolgozás}

A gyakorlatban előfordul, hogy a szállító ugyanazt a számlát több csatornán (e-mail, posta) is elküldi, vagy különböző ügyintézők dolgozzák fel ugyanazt a dokumentumot.
Amikor a számlaszámot kissé eltérően rögzítik (pl. ,,SZ-100'' helyett ,,SZ100''), a standard SAP szűrési mechanizmus nem képes felismerni a duplikációt.
Ennek eredményeképpen a vállalat ugyanazt a számlát kétszer fizeti ki, ami jelentős pénzügyi veszteséget okoz.
Tamaro (2025) szerint a vállalatok mintegy egyharmada küzd duplikált fizetések problémájával, ami a manuális folyamatok alacsony megbízhatóságára utal \cite{Caso-Hidden-Costs-of-Manual}.
\paragraph{Elveszett Számlák}

Az e-mail alapú feldolgozás során a számlák könnyen elveszhetnek: a spam mappába kerülhetnek, a hálózati meghajtó rossz mappájába menthetik őket, vagy egyszerűen ,,eltűnhetnek'' a túlterhelt e-mail fiókokban.
Ferro (2020) rámutat, hogy az e-mailes jóváhagyási rendszerben ,,az AP részleg soha nem lehet biztos abban, hogy egy számla hol tart a folyamatban'' \cite{Edenredpay-5-Problems}.
Az elveszett számlák nemcsak késedelmi kamatokhoz vezetnek, hanem megingatják a szállítói kapcsolatokat is, hiszen a szállítók idővel elégedetlenné válnak az állandó fizetési késések miatt.
\subsubsection{Időigényes Lépések és Szűk Keresztmetszetek (Költség- és Időproblémák)}

A minőségi problémák mellett a manuális folyamatok jelentős mennyiségi és pénzügyi terheket is rónak a vállalatokra.
\begin{enumerate}
\item \textbf{Manuális Adatrögzítés (FTE Költség)} A manuális adatrögzítés jelentése az egyik legnagyobb költségtényező.
Bár számlánként csupán 3-5 percnyi gépelésről van szó, ez nagy volumenű feldolgozás esetén teljes munkaidős pozíciók (FTE) megtartását igényli.
Baran (2025) számítása szerint egy évente 40.000 számlát feldolgozó vállalatnál a számlánkénti 5 perc több mint 3.300 óra munkaidőt jelent évente \cite{Ascendsoftware-Cost-of-Manual}.
Összehasonlításképp: egy AI-alapú megoldás (mint az SAP Document Information Extraction) ugyanezt a feladatot másodpercek alatt elvégezheti.
Parseur (2025) adatai szerint a manuális feldolgozás költsége 10-15 dollár számlánként terjed, míg az automatizált megoldások ezt 2-3 dollárra csökkenthetik \cite{Parseur-AI-Invoice-Processing}.
\item \textbf{Jóváhagyóra Várakozás (Átfutási Idő)} Az e-mail alapú jóváhagyási folyamat a leggyakrabban előforduló szűk keresztmetszet.
Ferro (2020) kiemeli, hogy a vezetők naponta átlagosan 100 e-mailt kapnak, amelyek között az aprobálási kérések könnyen elvesznek \cite{Edenredpay-5-Problems}.
Az e-mailes folyamat ,,fekete lyukként'' működik: nincs átláthatóság arról, hogy ki látta a számlát, mikor vette át, és mennyi időre van szüksége a döntéshez.
A várakozási idő tovább növekszik, amikor többszörös jóváhagyási körök szükségesek, és a felelősök nem ismerik a vállalati szabályokat.
Tamaro (2025) szerint a manuális feldolgozás átlagos átfutási ideje 14.6 nap, ami kritikusan lassú a modern üzleti környezetben \cite{Caso-Hidden-Costs-of-Manual}.
\item \textbf{Elvesztett Skontó (Pénzügyi Veszteség)} A lassú átfutási idő közvetlen pénzügyi következménye az elvesztett korai fizetési kedvezmények (skontók).
Amikor egy számla hetekig várakozik a jóváhagyásra, a vállalat nem tud élni a 2-5\%-os árengedményekkel, amelyek jelentős összegeket jelentenek éves szinten.
Parseur (2025) egy esettanulmányában egy kiskereskedelmi vállalat évi 50.000 dollárt fizetett késedelmi bírságokként, részben az elvesztett skontók miatt \cite{Parseur-AI-Invoice-Processing}.
Baran (2025) is hangsúlyozza, hogy a manuális folyamatok csökkentik a korai fizetési kedvezmények igénybevételének lehetőségét, ami közvetlen veszteséget jelent a vállalat számára \cite{Ascendsoftware-Cost-of-Manual}.
\item \textbf{Átláthatóság Hiánya (Stratégiai Gát)} A manuális folyamatok talán legsúlyosabb stratégiai hátránya az átláthatóság teljes hiánya.
A menedzsment nem látja valós időben, hogy mennyi a vállalat aktuális kötelezettségállománya (liability), mivel a számlák tucatjai ,,keringenek'' az e-mail postafiókokban és a megosztott mappákban.
Ferro (2020) pontosan fogalmaz: ,,Az AP részleg soha nem lehet biztos abban, hogy egy számla hol tart a folyamatban'' \cite{Edenredpay-5-Problems}. Ez az átláthatatlanság lehetetlenné teszi a pontos pénzügyi tervezést, megnehezíti a cash flow menedzsmentet, és akadályozza a pénzügyi osztályt abban, hogy stratégiai partnerként funkcionáljon az üzleti döntéshozatalban.
\end{enumerate}

Összefoglalóan, a manuális számlafeldolgozás nem csupán hatékonysági problémát jelent, hanem komoly pénzügyi kockázatokat és stratégiai akadályokat is.
A fenti problémák együttesen indokolják a \textbf{\ref{chap:megvalositas}}. fejezetben bemutatásra kerülő hiperautomatizálási megoldás megvalósításának szükségességét, amely célzottan ad választ ezekre a kihívásokra.
\subsection{Fejezet Összegzése és Átvezetés a Megoldásra}

A harmadik fejezet átfogó elemzése egyértelműen rámutat a szállítói számlafeldolgozás modern vállalati gyakorlatában rejlő fundamentális szakadékra.
Egyrészt bemutattuk az SAP rendszer által kínált ideális, ,,tankönyvi'' folyamatot, amely a \textbf{háromoldalú egyeztetés} elvén, a \textbf{PO-alapú (MIRO)} és \textbf{nem PO-alapú (FB60)} számlák szisztematikus feldolgozásán, valamint a \textbf{beépített automatizálási eszközökön} – mint például a \textbf{toleranciahatárok}, az \textbf{ERS} és az \textbf{EDI/IDoc} – keresztül magas fokú hatékonyságot, kontrollt és biztonságot kínál.
Ez a modell elméletileg képes a humán hibák minimalizálására és a folyamatok optimális lebonyolítására.
Másrészt, a valóságot vizsgálva – a tipikus vállalati gyakorlatot (,,As-Is'') – egy lényegesen kevésbé hatékony és kockázatosabb kép rajzolódott ki.
Itt a technológiai keretrendszer és a tényleges működés között óriási rés tátong.
A folyamat kritikus szakaszai – a számlák \textbf{beérkeztetése (intake)}, a manuális \textbf{adatrögzítés} (,,swivel-chair integration''), valamint a belső \textbf{jóváhagyási körök} – túlnyomórészt az SAP-on kívül, e-mail alapú, strukturálatlan és erőforrás-igényes manuális tevékenységekbe torkollnak.
Ez a hibrid modell számos súlyos problémát generál, melyeket a \textbf{\ref{sec:problemfeltaras_hibai_koltsegei}}.
pont részletesen feltárt:

\begin{itemize}
\item \textbf{Minőségi kockázatok:} Megbízhatatlan adatátvitel, \textbf{adathibák}, \textbf{duplikált fizetések} és \textbf{elveszett számlák}, melyek közvetlen pénzügyi veszteségekhez vezetnek.
\item \textbf{Operatív és pénzügyi költségek:} A jelentős \textbf{manuális munkaerő-felhasználás (FTE költség)}, a jóváhagyási csúcsok miatti extrém \textbf{átfutási idők}, az \textbf{elvesztett korai fizetési kedvezmények (skontók)} és a késedelmi kamatok.
\item \textbf{Stratégiai akadályok:} A folyamat teljes \textbf{átláthatatlansága}, ami lehetetlenné teszi a valós idejű kötelezettségállomány nyomon követését és alátámasztja a pénzügyi tervezés megbízhatóságát.
\end{itemize}

Összefoglalóan, míg az SAP rendszer technikai képességei adottak a hatékony feldolgozáshoz, a valós vállalati gyakorlatot jellemző, az SAP-ra épülő, de attól független manuális ,,párhuzamos univerzum'' ellehetetleníti ezek kihasználását.
Ez a szakadék, illetve a fent felsorolt konkrét problémák adják a szakdolgozat szilárd \textbf{üzleti indoklását (Business Case)}.
\textbf{Ez az indoklás vezet át közvetlenül a \textbf{\ref{chap:megvalositas}. fejezetbe}}, ahol a problémafelvetésből kiindulva egy konkrét, célzott megoldást mutatunk be.
A \textbf{\ref{chap:megvalositas}. fejezetben} egy olyan prototípus tervezését és megvalósítását fogjuk részletezni az \textbf{SAP Build Process Automation (SBPA)} eszközeivel, amely pontosan a \textbf{\ref{sec:problemfeltaras_hibai_koltsegei}}.
pontban azonosított gyengeségekre fókuszál. A prototípus célja, hogy intelligens automatizálással – beleértve a robotizált folyamatautomizálást (RPA) és a dokumentumfeldolgozó AI (Document Information Extraction) szolgáltatásait – megszüntesse a manuális adatrögzítés szűk keresztmetszetét, integrált workflow-val pótolja az e-mailes jóváhagyási ,,fekete lyukat'', és ezáltal radikálisan csökkentse a feldolgozási időt, a költségeket és a hibák arányát, miközben visszanyeri a folyamat teljes átláthatóságát.